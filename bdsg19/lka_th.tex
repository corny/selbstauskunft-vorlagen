auf der Grundlage von §~19 I Bundesdatenschutzgesetz (BDSG),
§~13 Thüringer Datenschutzgesetz (ThürDSG) sowie
§~47 Thüringer Gesetz über die Aufgaben und Befugnisse der Polizei (ThürPAG) bitte
ich Sie um Auskunft zu folgenden Punkten:

\begin{itemize}
  \item über die durch Ihre Behörde zu meiner Person in Systemen der elektronischen
  Datenerfassung und -verarbeitung gespeicherten Daten, im Besonderen auch über
  in Datenbanken zur Vorgangsbearbeitung wie IGVP vorgehaltene Daten;

  \item über den Zweck der Verarbeitung;

  \item über die Herkunft der Daten, soweit diese gespeichert oder sonst bekannt ist;

  \item über die Empfänger oder die Gruppen von Empfängern, an die die Daten übermittelt wurden.
\end{itemize}

Meiner Anfrage liegt ein generelles Informationsinteresse unter Wahrnehmung
meines verfassungsrechtlich verbürgten Grundrechts auf informationelle
Selbstbestimmung zugrunde.

Vorsorglich weise ich darauf hin, dass ich dies als hinreichend im
Hinblick auf §47 I ThürPAG ansehe. Da mir kaum stichhaltigere Gründe
für ein Auskunftsverlangens einfallen als eben die Wahrnehmung eines
Grundrechts, werde ich keine weiteren Gründe angeben.
